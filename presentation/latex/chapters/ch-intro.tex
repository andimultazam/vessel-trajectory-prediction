\SetPicSubDir{ch-Intro}

\chapter{Introduction}
\vspace{2em}

\section{Overview}

Singapore maintains its position as the leading maritime capital in the world for being tops in three pillars including Shipping, Ports and Logistics, Attractiveness and Competitiveness, and for the remaining pillars, it is within the top 10 cities \cite{dnvgl2019maritime}. The award has been received for four consecutive times by The Republic since 2012 when it was first published \cite{straittimes2019maritime}. As Singapore continues to become one of the world's busiest port in the world, the Maritime Port Authority (MPA) needs to enhance maritime security at Singapore water against dangerous adversaries such as piracy attacks, armed robbery, crew abduction, or smuggling. The Regional Cooperation Agreement on Combating Piracy and Armed Robbery against Ships in Asia (ReCAAP) recorded the number of sea attacks in Singapore Strait jumped from 8 incidences in 2018 to 31 incidents in 2019. Because of the increase of incidents, ReCAAP recommends to the law enforcement agencies to enhance surveillance, increase patrols and respond promptly to incidents reported by ships while ships master and crew are advised to exercise enhanced vigilance when transiting the area of concern \cite{recaap2019report}.

The challenges of sea surveillance in Singapore water are due to its size and vessel movements traffic. The Singapore Straits is extended for 105-km long from the Strait of Malacca in the west to the South China Sea in the east, and 16-km wide lies between Singapore Island in the north to Riau Island of Indonesia in the south. Nearly 100,000 vessels pass through the 105-km long waterway each year, accounting for about a quarter of the world's trade good \cite{scmp2019maritime}. Our system records the average of 1 million vessels message in a day, leading to about 800 vessels information to keep track per minute. Those number makes it challenging for coastal police guard to patrol the vessels only when the accident takes place, hence there is a need of technology to predict the future location of vessels based on their history of movements. Some predictions would allow the police to take necessary action in advance and prevent the incidence from happening.

A standard way of sending information about ship movement is through the Automatic Identification System (AIS). Since 2002, International Maritime Organization required every ship of over 300 gross tonnages engaged in an international voyage, and passenger vessels irrespective of their size to be installed with AIS transceivers that allow a vessel to broadcast information automatically about their location in the sea. The AIS message that can be received by a nearby station equipped with AIS receiver module varies from every 10 seconds to 6 minutes. 

The primary objectives of this thesis are 2 folds; first is to understand the statistics of AIS data, extract some patterns, and gain valuable insight from it. Secondly, we examined several techniques of vessel trajectory prediction using machine learning and historical data. We focus our study on AIS data of vessels crossing the Singapore Straits in December 2019.

\section{Research Objectives}
In this thesis, we seek to answer the following research questions:

\begin{itemize}
    \item Can we gain a better understanding of AIS big data using statistics and visualization by exploiting some of their properties such as coordinates, speed, bearing, vessel ID, vessel country origin, vessel type, and destination over time?
    \item Can we predict the future movement of vessels based upon AIS historic information? How can we propose a prediction model to do so?
\end{itemize}

\section{Thesis Synopsis}

The rest of this thesis is organized as follows. 
In \autoref{ch:review}, we explain about AIS message in detail, conduct a literature review related to past works in trajectory analysis in different domain and maritime fields using AIS data, explain our end-to-end architecture system behind AIS data generation, review the state of the art of artificial neural network for sequential data.
\autoref{ch:eda} describes the AIS data from a statistical point of view with visualization. We discuss AIS features and the message it entails, perform some trajectory analysis, and identify erroneous values across all features.
\autoref{ch:univariate} provides a methodology of designing prediction models, data pipeline, data cleaning, feature selection, and the rationale behind model hyper-parameter choice.
In \autoref{ch:noodle} we design and implement an experiment to find out under what condition the prediction model can suggest a better performance.
We conclude the entire thesis as well as discuss further directions for future research in \autoref{ch:concl}.